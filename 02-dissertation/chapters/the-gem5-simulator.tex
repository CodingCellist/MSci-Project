\section{Overview}
The gem5 Simulator \cite{binkert_gem5_2011} is a state-of-the-art hardware 
simulator used not only for architecture research, but also by the companies 
which develop the hardware it simulates. ARM, for example, have detailed their 
use of it at the ARM research summit in 2017. It is the result of merging two 
simulators in 2011: the m5 simulator, which had great support for multiple 
hardware architectures and ISAs, and the GEMS simulator, whose main focus was 
memory simulation and hence allowed for detailed memory models and various 
cache coherence protocols \cite{binkert_gem5_2011}.

Due to the enormous complexity of such a piece of software, the way to start 
using gem5 is to download and build from source \cite{noauthor_gem5_nodate-2}. 
Building the simulator can be a bit complicated, due to it being sensitive to 
various system tools and configurations. The gem5 Simulator uses \texttt{scons} 
as its build tool. Specifically, it seems to rely on the Python 2 version of 
\texttt{scons} (and also uses Python 2 for its other Python scripts) despite 
Python 2 having been officially deprecated on the 1\textsuperscript{st} of 
January 2020, with the move to do so being announced well in advance 
\cite{noauthor_sunsetting_nodate}. However, this can be easily fixed by using a 
Python 2 virtual environment. The specific version of the GNU Compiler 
Collection (GCC) also seems to affect things. I was using GCC version 9.3.0 
which must have more warnings than the version used by the gem5 developers, as 
it failed to build gem5 due to the \texttt{-Werror} flag being turned on. This 
flags turns compiler warnings, which normally highlight suspicious areas of 
code but still lets the build go ahead, into errors which do not let the build 
go ahead. The user can tell the compiler that some warnings are exceptions to 
the \texttt{-Werror} flag, but due to the number of warnings I was getting, it 
was simpler to remove the flag from the \texttt{scons} configuration found in 
the \texttt{SConstruct} file. There are also a number of packages that the 
build script might recommend one installs, most notably the Google Performance 
Tools (gperftools) \cite{noauthor_google_nodate} which they suggest improves 
performance by 12\%. Once the necessary software, packages, etc. have been 
installed and configured, the build process takes around 20-30 minutes when 
building using 13 threads on an 8\textsuperscript{th} generation Intel Core i7 
processor, which also says something about the scale and complexity of gem5.

\section{Configuration and Setup}
There are two parts of gem5 that can be configured: the simulator binary and 
its configuration files. The binary produced by \texttt{scons} supports a 
number of flags which allow the user to specify things like the output 
directory, whether to redirect gem5's standard output and/or standard error 
streams, and what to call various files being output. There are also a number 
of debug flags for developing gem5 itself. The most important of the flags is 
arguably the \texttt{--outdir} flag used for specifying the output directory. 
Without specifying the output directory when running multiple simulations side 
by side, if no directory is specified, i.e. the default \texttt{m5out} 
directory is used, each instance will try to write to the same files as the 
others, resulting in non-deterministic race conditions in terms of which 
process was the last to write to the file, leading to unusable, interweaved 
output.

\begin{itemize}
    \item explain how to configure it, both the gem5.opt binary itself and the
          python scripts
    \item explain the layouts of the various python scripts, the need to
          customise them, and what changes were made (provide code examples!)
    \item as part of the above, explain the various features that are needed for
          this dissertation and why
    \item mention cross-compilation, what it is, why it is needed, and how we
          got it working
\end{itemize}