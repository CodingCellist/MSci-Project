Accurate full system simulators, with support for the latest hardware and its 
features are hugely complex, both for the developers and the users. The gem5 
simulator is definitely a powerful tool, but it unfortunately seems to be in a 
somewhat fragile and undocumented state concerning DVFS and power models at the 
moment.

Whilst the predictions done were not the most precise, it was demonstrated that 
predictions can be done. Changing the configurations was shown to affect the 
performance and energy consumption of the setups, sometimes drastically. This 
demonstrates the importance of managing the DVFS of cores and clusters, and that
there is potentially a lot to be gained in terms of power savings. Based on the 
results from the predictions and the baselines, it seems schedulers should be 
able to use the PMUs to make informed decisions as to what silicon to keep dark 
in order to balance energy consumption and performance.

Given more time, I would have liked to get more data by also running 
configurations with only big or LITTLE cores and try to get the predictions to 
work for varying numbers of threads as well. Other future work could include 
the implementation on real hardware with multiple programs running 
simultaneously; improving the predictions by determining the feature importance 
of the different PMU events; playing with the balance of the optimum finder, to 
see if better optima could be found if power is prioritised over number of 
cycles, or vice versa.
