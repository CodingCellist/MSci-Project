\section{Processes and Threads}
The scheduling problem has been around for a while. The earliest papers which
discuss scheduling date back to 1962 \cite{corbato_experimental_1962}, with
hard real-time scheduling (i.e. scheduling for systems where time limits
\textit{must} be met) dating back to 1973 \cite{liu_scheduling_1973}, and
preemptive scheduling being discussed in 1975 \cite{kleinrock_computer_1976}. In
this section I will give an overview of the scheduling problem, explaining some
key concepts and problems, basing a lot on Silberschatz et al.'s book on
operating systems \cite{silberschatz_operating_2014}.

Code running is represented as a \textit{process}. Processes contain the code
being run, information such as what process started the current process (its
\textit{parent}), what process(es) the current process has created (its
\textit{children}), the state of the process (e.g. `running', `waiting',
`ready'), and also some information about the current memory and scheduling
information about the process's priority, how long it has been running for, etc.
A process may contain multiple \textit{threads}. A thread is a simple unit of
computation or work being done. A thread has an ID and contains a stack, a
register set, and a program counter. The other parts, e.g. the code or memory
information, is stored in the process running the thread and is shared between
all the process's threads. Hence, creating new threads and inter-thread
communication is cheaper than creating new processes and inter-process
communication. This is useful for processes which want to do different, often
independent things in parallel, e.g. a web browser fetching data from the
internet whilst also handling the user's keystrokes. It can also be useful if
the problem being solved can be broken up into smaller parts which can then be
done in parallel.

In most major operating systems, MacOS, Windows, Solaris, and Linux, there are
both user threads and kernel threads. Kernel threads, as the name implies, are
managed directly by the operating system kernel. User threads, on the other
hand, are created and managed by the user/programmer. Kernel threads are more
expensive to create than user threads and as such, most operating systems map
user threads to kernel threads during runtime. This means that the
user/programmer does not have to worry about how many threads they create, and
that the operating system can reuse kernel threads for various applications,
thereby improving performance. When it comes to scheduling, the scheduler
manages kernel threads.

\section{Scheduling}
When talking about scheduling, the terms ``process scheduling'' and ``thread
scheduling'' are often used interchangeably. To avoid confusion with the
previous section and to be consistent with commonly used Linux terminology, I
shall use ``task'' to refer to anything, be it a process or a thread, which
needs to be scheduled. Scheduling is the problem determining what tasks need to
run and for how long. Most tasks do not require constant computation. Instead,
computation happens in bursts and the other time is, for example, spent on
waiting for I/O to happen, e.g. reading a value from memory. So in order to
maximise CPU usage, a scheduler swaps different tasks on the CPU(s). Another
reason scheduling is required, is that if the scheduler did not swap the tasks,
a single CPU-intensive task could hog the CPU for a long time. On interactive
systems, this could result in loss of interactivity (aka. ``freezing'') until
the heavy task finishes its computations. By swapping tasks, the scheduler tries
to keep the CPU-usage as fair as possible. Ideally, the perfect scheduler would
let the task that has the least time left run first, thereby minimising the wait
time for the other tasks. However, this would require the scheduler to know the
future (i.e. when the various tasks would stop) and is therefore unfortunately
impossible. Instead, there exist various scheduling algorithms which are used to
determine what task to run next. One way to determine this is to assign the
tasks a priority and execute the highest-priority task first. This works fine if
the high-priority jobs finish. However, there is a very real risk that a
low-priority job may never be run as it is kept at the back of the queue by
higher-priority jobs. As such, priority scheduling is rarely used in its
na{\" i}ve version, with modern scheduling algorithms changing the priority of a
task over time, based on various variables. One example of this is the Linux
``Completely Fair Scheduler'' (CFS) which was introduced in version 2.6.23
\cite{noauthor_cfs_nodate}.

    \subsection{The Completely Fair Scheduler}
    Typically, the tasks that are waiting to be executed on the CPU are stored
    in a so-called ``ready queue''. However, the CFS stores the tasks in a
    data structure known as a Red-Black tree (RB-tree). RB-trees were first
    introduced in 1978 by Guibas and Sedgewick \cite{guibas_dichromatic_1978}.
    An RB-tree is a type of self-balancing binary tree. In addition to having
    the usual binary tree attributes (a key, the left children being less than
    their parent, and the right children being greater than their parent), each
    node in the tree is given a colour, red or black, and the tree balances
    itself by maintaining the following three properties:
    \begin{enumerate}
        \item The root of an RB-tree is black.
        \item The children of a red node are black.
        \item The paths going from the root to a \texttt{null} leaf all contain
              the same number of black nodes.
    \end{enumerate}
    A complete overview of how the operations on RB-trees work is beyond the
    scope of this dissertation. The important attribute is that, like other
    self-balancing trees, an RB-tree maintains a height that is very close to
    (or exactly) $log N$, regardless of how many operations have been performed,
    allowing lookup to stay $O(log N)$. By keeping the tasks in an RB-tree, the
    CFS can access them quickly while the data structure makes sure to permute
    the tasks according to their priority. To further increase performance, the
    CFS caches a pointer to the leftmost node in RB-tree.
    \\
    
    The CFS distinguishes between real-time tasks and `normal' tasks. Real-time
    tasks have a regular priority and are scheduled accordingly. Normal tasks
    each have a virtual runtime (\texttt{vruntime}) and a ``nice value''. The
    \texttt{vruntime} is used as the key for the RB-tree storing the tasks and
    the CFS schedules the leftmost leaf in the RB-tree. The nice value is an
    integer between $-20$ and $+19$ which affects the recorded \texttt{vruntime}
    for that task. A task's \texttt{vruntime} is equal to its actual/physical
    runtime (the time it spent on the CPU) and some modifier based on the task's
    nice value. A negative nice value leads to a lower \texttt{vruntime} than
    actual runtime, a nice value of 0 to an equal value, and a positive nice
    value to a greater \texttt{vruntime} than the actual runtime. This means
    that a task can ``be nice'' to other tasks by setting its nice value higher,
    resulting in a bigger \texttt{vruntime}, leading it to be scheduled less
    often than other tasks. And conversely, tasks with a low nice value will be
    scheduled more often despite them potentially taking up more physical
    runtime. All tasks eventually get to run, as a task's \texttt{vruntime} can
    only increase, meaning that even if it does so slowly (due to a low nice
    value) it will eventually be greater than a task whose \texttt{vruntime} has
    not changed because it was not being scheduled. This task will then be the
    leftmost node, and so it will get scheduled.

The problem of how to schedule the various
jobs running on a computer is complex, and has only gotten more complex as
multicore processors (i.e. chips with multiple CPU cores on them) became
commonplace. With the introduction of simultaneous multithreading, the ability
to run more than one thread per CPU core, the problem has gotten further
complex. \todo{mention asymmetric single-ISA}


\begin{itemize}
    \item mention that the problem has gotten more complex as we now have
          multicore processors and simultaneous multithreading (explain these)
\end{itemize}

\section{ARM big.LITTLE architecture}
\begin{itemize}
    \item what is an ISA?
    \item what is single-ISA means and why it is helpful?
    \item mention that ARM big.LITTLE is an example of this
    \item what is a big core and what is a LITTLE core?
    \item why do we have this split?
    \item mention that this poses new challenges to the already complex problem
          that scheduling is
\end{itemize}
